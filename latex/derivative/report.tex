\documentclass[a4paper,11pt]{article}

\usepackage[utf8]{inputenc}

\usepackage{minted}
\usepackage{hyperref}

\usepackage{amsmath}

\begin{document}

\title{
    \textbf{Derivative}
}
\author{Hannes Mann}
\date{2023-01-22}

\maketitle

\section*{Introduction}

This is a report for the first assignment in the course ID1019 Programming II.
In this assignment, a program should be created that can take the derivative of a mathematical expression and present it in a simplified form.

The implementation was done in Elixir. No build system has been set up at the time the report is written, but the script can be executed with \mintinline{elixir}{elixir main.exs}.

The source code can be found at: \href{https://github.com/hannesmann/ID1019/tree/main/src/derivative}{GitHub}.

\section*{Implementing differentiation rules}

For the program to be able to manipulate mathematical expressions it needs to represent the expression in memory.
Working with strings would get complicated very quickly so a tuple type was created that can hold all necessary operations and values.

The operations were chosen based on how they apply to derivatives, while $x^n$ can be represented as $x*x*x*x\dots$ and $cos(x)$ can be represented as $sin(90-x)$ it's useful to
separate them because they have different behavior in the program.

\begin{minted}[tabsize=2,fontsize=\footnotesize,breaklines]{elixir}
@type literal() :: { :num, number() } | { :var, atom() }

@type expr() ::
  # a + b
  { :add, expr(), expr() } |
  # a * b
  { :mul, expr(), expr() } |
  # a / b
  { :div, expr(), expr() } |
  # a ^ n
  { :pow, expr(), { :num, number() } } |
  # ln(a)
  { :ln, literal() } |
  # sin(a)
  { :sin, literal() } |
  # cos(a)
  { :cos, literal() } |
  literal()
\end{minted}

The program uses pattern matching on the operation type to figure out what differentiation rule to use.
Several functions are defined with the same name and Elixir will choose the right one depending on the type.

If we are deriving with respect to $x$ the program also needs to differentiate between the variable $x$ and other variables (which are assumed to be constant).
The derivative function is passed the name of the variable so that if the variable is $x$, the result is 1, otherwise the result is 0.

\begin{minted}[tabsize=2,fontsize=\footnotesize,breaklines]{elixir}
  # The variable in the expression is the same variable as v
  # d/dv v = 1
  def deriv({ :var, v }, v) do { :num, 1 } end
  # The variable in the expression is an unrelated constant
  # d/dv c = 0
  def deriv({ :var, _ }, _) do { :num, 0 } end
\end{minted}

Some functions will call \mintinline{elixir}{deriv} recursively which is how the program is able to derive an expression consisting of several terms.

\begin{minted}[tabsize=2,fontsize=\footnotesize,breaklines]{elixir}
  def deriv({ :add, e1, e2 }, v) do
    { :add, deriv(e1, v), deriv(e2, v) }
  end

  # Product rule
  def deriv({ :mul, e1, e2 }, v) do
    { :add, { :mul, deriv(e1, v), e2 }, { :mul, e1, deriv(e2, v) } }
  end

  # Quotient rule
  def deriv({ :div, e1, e2 }, v) do
    { :div, { :add, { :mul, deriv(e1, v), e2 }, { :mul, { :mul, e1, deriv(e2, v) }, { :num, -1 } } }, { :pow, e2, { :num, 2 } } }
  end
\end{minted}

\section*{Simplifying expressions}

When calling \mintinline{elixir}{deriv} on expressions consisting of several terms the program will often output an expression consisting of ''useless'' terms like adding zero or multiplying $x$ by zero.
This is because the function never eliminates expressions without side effects. It only goes through each term and derives them in order.
To simplify expressions I defined four rules:
\begin{itemize}
	\item \textbf{Constant folding}: Any operation consisting of two constant terms can be evaluated and transformed into \mintinline{elixir}{{:num, n}} (example: $2+3 = 5$, $2^3 = 8$).
	\item \textbf{Constant folding inside parentheses}: The same rule can be applied if we have one constant outside parentheses and one inside (example: $2+(3+x)=5+x$, $2*(3*x) = 6*2x$).
  \item \textbf{Statements without side effects}: Addition by zero, multiplication by one, division by one, $x$ to the power of one.
  \item \textbf{Statements that always evaluate to the same value}: Multiplication by zero, $x$ to the power of zero, zero to the power of $x$.
\end{itemize}

This covers most (but surely not all) cases where an expression is unnecessarily complicated.
These four rules are easy to implement with pattern matching with Elixir,
but to solve all cases it seems the program would have to run multiple passes and keep track of
constants and variables on a stack.

\section*{Result}

The program includes some example expressions to test the derivative and simplification function.
Expressions are printed without any parentheses and the program cannot know implicitly when to insert them
since this would require writing code that has knowledge of operator precedence and that was not the focus of this exercise.

\begin{minted}[tabsize=2,fontsize=\footnotesize,breaklines]{elixir}
Expression: 2*x^3
Derivative: 6*x^2
Expression: sin(x)
Derivative: cos(x)

# From course example
Expression: 4*x^2+3*x+42
Derivative: 8*x+3

# x^0.5 is printed as sqrt(x)
Expression: sqrt(x)
Derivative: 0.5*x^-0.5
\end{minted}

\end{document}
