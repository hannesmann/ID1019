\documentclass[a4paper,11pt]{article}

\usepackage[utf8]{inputenc}

\usepackage{minted}
\usepackage{hyperref}

\usepackage{amsmath}

\begin{document}

\title{
    \textbf{Advent of Code 2022 Day 7}
}
\author{Hannes Mann}
\date{2023-02-17}

\maketitle

\section*{Introduction}

This is a report for the fifth assignment in the course ID1019 Programming II.
In this assignment, one of the exercises from Advent of Code should be solved. In this report the puzzle that was chosen is \href{https://adventofcode.com/2022/day/7}{Day 7: No Space Left On Device}.

The implementation was done in Elixir. A bash script is included in the repository that can be used to run the program: \mintinline{text}{./run-project.sh advent07}.

The source code can be found at: \href{https://github.com/hannesmann/ID1019/tree/main/src/advent07}{GitHub}.

\section*{Determining the size of directories}

The first part of the exercise is to parse a list of commands (simulating a person typing in a Unix terminal) and create a filesystem structure.
The filesystem should then be used to determine the size of all directories which contain less than 10000 bytes.

\subsection*{The parser}

The list of commands will look something like this:
\begin{minted}[tabsize=2,fontsize=\footnotesize,breaklines]{sh}
$ ls
dir a
593243 b
$ cd a
$ ls
dir c
dir d
100 e
$ cd ..
\end{minted}

The $\$$ symbol looks like a shell prompt (from bash, zsh, etc...) but it can actually be used to distinguish between commands and directory listings.
A command will always be \mintinline{text}{cd} or \mintinline{text}{ls} and a directory listing will contain commands \mintinline{text}{dir} and \mintinline{text}{[size]}
where \mintinline{text}{[size]} says that the entry is a file with the specified size.

Handling \mintinline{text}{cd} is done by keeping a stack that we can push (if we move into a directory) and pop (if we move back to the parent).
We use \mintinline{text}{reduce} in Elixir to keep track of state.

\begin{minted}[tabsize=2,fontsize=\footnotesize,breaklines]{elixir}
case Enum.at(cmd, 1) do
  "cd" ->
    [_, cmd, extra] = String.split(String.trim(row), " ")

    case extra do
      "/" -> {root, []}
      ".." -> {root, Enum.drop(stack, -1)}
      _ -> {root, stack ++ [extra]}
    end
  _ -> {root, stack}
end
\end{minted}

Filesystem entires are added to \mintinline{text}{root} with recursion.
The function walks down the stack until it reaches the current directory and adds either a directory or file entry.

\begin{minted}[tabsize=2,fontsize=\footnotesize,breaklines]{elixir}
# adddir works the same way
def addfile({name, entries}, [], file, size) do
  {name, entries ++ [{file, size}]}
end
def addfile({name, entries}, [head | tail], file, size) do
  {name, Enum.map(entries, fn {n, l} ->
    if head == n do
      addfile({n, l}, tail, file, size)
    else
      {n, l}
    end
  end)}
end
\end{minted}

\subsection*{The filesystem}

The finished filesystem structure will contain entries which are either files \mintinline{text}{{name, size}} or directories \mintinline{text}{{name, entries}}.
Starting at the root we can walk down the filesystem and find all directories that are big enough to qualify:

\begin{minted}[tabsize=2,fontsize=\footnotesize,breaklines]{elixir}
def dirs_less_than_100000(entry, acc \\ [])
def dirs_less_than_100000({name, entries}, acc) when is_list(entries) do
  this_size = size({name, entries})

  if this_size <= 100000 do
    Enum.reduce(entries, acc ++ [{name, entries}], &dirs_less_than_100000/2)
  else
    Enum.reduce(entries, acc, &dirs_less_than_100000/2)
  end
end
def dirs_less_than_100000(_, acc) do acc end
\end{minted}

Once we have a list of all directories we can sum them together with \mintinline{text}{reduce}:

\begin{minted}[tabsize=2,fontsize=\footnotesize,breaklines]{elixir}
Enum.reduce(dirs, 0, fn e, acc -> acc + Filesystem.size(e) end)
\end{minted}

\section*{Determining what directory to delete}

The second part of the exercise is determining what directory we should delete to free up enough space for the system update to go through.
The total disk space is 70000000 bytes and we need at least 30000000 bytes to perform the update.

\subsection*{Determining the amount of free space on the disk}

To determine the amount of free space on the disk we need to know two things: How big the disk is and how much space all the files take up.
We already know how big the disk is and we can measure the size of the root directory to find out how much space all the files take up together.

\begin{minted}[tabsize=2,fontsize=\footnotesize,breaklines]{elixir}
root_size = Filesystem.size(root)
free_space = 70000000 - root_size
required_to_delete = 30000000 - free_space
\end{minted}

\subsection*{Determining the smallest viable directory}

Now that we know how many bytes we need to delete we should find the smallest directory that is still big enough to free up enough space.

To find all the directory in the filesystem we only need to start at the root directory and walk down the tree, ignoring any files.
The condition to determine if something is a file or directory is checking if \mintinline{text}{entries} is a list.
In hindsight, giving every entry a unique atom would have made pattern matching much easier.

\begin{minted}[tabsize=2,fontsize=\footnotesize,breaklines]{elixir}
# Walk tree and find all dirs
def dirs(entry, acc \\ [])
def dirs({name, entries}, acc) when is_list(entries) do
  Enum.reduce(entries, acc ++ [{name, entries}], &dirs/2)
end
def dirs(_, acc) do acc end
\end{minted}

Now that we can obtain a list of all directories we only need to sort it by size and filter out any directories that are too small:
\begin{minted}[tabsize=2,fontsize=\footnotesize,breaklines]{elixir}
sorted_by_size = Enum.filter(dirs,
  fn e -> Filesystem.size(e) >= required_to_delete end
)
sorted_by_size = Enum.sort_by(sorted_by_size, &Filesystem.size/1)
{to_delete, _} = List.first(sorted_by_size)
\end{minted}

\section*{Result}

Running the program on the puzzle input (included in src/advent07/day07.txt) gives the following results:

\begin{minted}[tabsize=2,fontsize=\footnotesize,breaklines]{text}
Sum of directories with total size <= 100000: 1444896

root_size: 40389918
free_space: 29610082
required_to_delete: 389918

Directory to be deleted: mdclfbs
Size: 404395
\end{minted}

\end{document}
