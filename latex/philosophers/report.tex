\documentclass[a4paper,11pt]{article}

\usepackage[utf8]{inputenc}

\usepackage{minted}
\usepackage{hyperref}

\usepackage{amsmath}

\begin{document}

\title{
    \textbf{Philosophers and Concurrency}
}
\author{Hannes Mann}
\date{2023-02-24}

\maketitle

\section*{Introduction}

This is a report for the last higher-grade assignment in the course ID1019 Programming II.
In this assignment, a system should be implemented with concurrency that simulates philosophers sitting around at a table sharing a limited number of chopsticks.

The implementation was done in Elixir. A bash script is included in the repository that can be used to run the program: \mintinline{text}{./run-project.sh philosophers}.

The source code can be found at: \href{https://github.com/hannesmann/ID1019/tree/main/src/philosophers}{GitHub}.

\section*{Actors in the program}

The program is written with concurrency in Elixir, which means it uses lightweight processes to represent different actors. The program has two main "actors":

\begin{itemize}
	\item \textbf{Philosopher}: A philosopher will either be dreaming (sleeping), waiting for two chopsticks or eating. Once the philosopher has eaten it will not request the chopsticks again.
	\item \textbf{Chopsticks}: A chopstick can either be available or gone. This acts like a lock, in that one philosopher can only hold a chopstick at any one time. It can only be released by the philosopher that borrowed it.
\end{itemize}

As the program starts it will spawn $n$ chopsticks and $m$ philosophers. Since every philosopher needs two chopsticks to eat, if $n \leq m$ every philosopher except one will be required to wait at some point.

Just from reading the description it's obvious that a deadlock can occur. Since the philosophers will wait for a chopstick forever, two philosophers can get stuck waiting on the same chopstick forever.
The first version of the philosopher looked like this:

\begin{minted}[tabsize=2,fontsize=\footnotesize,breaklines]{elixir}
def rand_sleep(0) do :ok end
def rand_sleep(t) do
  # Average: :rand.uniform(t)
  :timer.sleep(:rand.uniform(t) * 2)
end

def start(hunger, left, right, name, ctrl) do
  spawn_link(fn -> dreaming({hunger, left, right, name, ctrl}) end)
end

def dreaming({hunger, left, right, name, ctrl}) do
  # Sleep for a while...
  rand_sleep(200)
  # Wake up!
  waiting({hunger, left, right, name, ctrl})
end

def waiting({hunger, left, right, name, ctrl}) do
  Chopstick.request(left)
  IO.puts("[#{name}] received left chopstick")
  Chopstick.request(right)
  IO.puts("[#{name}] received right chopstick")

  eating({hunger, left, right, name, ctrl})
end

def eating({hunger, left, right, name, ctrl}) do
  Enum.each(0..hunger, fn _ ->
    # Eating...
    rand_sleep(100)
  end)

  Chopstick.return(left)
  Chopstick.return(right)

  IO.puts("[#{name}] finished eating")
  send(ctrl, :done)

  dreaming({hunger, left, right, name, ctrl})
end
\end{minted}

If we wait between requesting the left and right chopstick, a deadlock will occur every time.
To resolve this we will request the chopsticks at the same time and we will add a timeout to the philosopher. If the philosoper has waited too many times for a chopstick it dies:

\begin{minted}[tabsize=2,fontsize=\footnotesize,breaklines]{elixir}
# Returns immediately
def send_request(stick) do
  send(stick, {:request, self()})
end

# This gives the philosoper a function it can choose to call
def get_wait_function(stick, timeout) do
  fn ->
    receive do
      {:received, stick} -> :ok
    after timeout ->
      :no
    end
  end
end

# Strength reaches 0
def waiting({_, _, _, name, ctrl, 0, _}) do
  IO.puts("[#{name}] died waiting for chopstick")
  send(ctrl, :done)
end

def waiting({hunger, left, right, name, ctrl, strength, {left_waiter, right_waiter}}) do
  case {left_waiter.(), right_waiter.()} do
    {:ok, :ok} -> eating({hunger, left, right, name, ctrl, strength})
    {_, :ok} -> waiting({hunger, left, right, name, ctrl, strength - 1, {left_waiter, fn -> :ok end}})
    {:ok, _} -> waiting({hunger, left, right, name, ctrl, strength - 1, {fn -> :ok end, right_waiter}})
    _ -> waiting({hunger, left, right, name, ctrl, strength - 2, {left_waiter, right_waiter}})
  end
end
\end{minted}

\section*{Benchmarks}

We can first try to spawn $n$ philosophers and $n$ chopsticks:
\begin{minted}[tabsize=2,fontsize=\footnotesize,breaklines]{elixir}
n=10 - 1877 ms
n=20 - 2062 ms
n=30 - 2341 ms
n=40 - 2307 ms
\end{minted}

As we can see the time does not increase as we spawn more philosophers. It's only when we limit the number of chopsticks to five that the program takes longer:
\begin{minted}[tabsize=2,fontsize=\footnotesize,breaklines]{elixir}
n=10 - 2110 ms
n=20 - 4474 ms
n=30 - 6959 ms
n=40 - 9579 ms
\end{minted}

The time starts growing exponentially, as the number of philosophers stuck in timeout that eventually die increase.

\end{document}
