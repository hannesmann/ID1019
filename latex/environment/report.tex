\documentclass[a4paper,11pt]{article}

\usepackage[utf8]{inputenc}

\usepackage{minted}
\usepackage{hyperref}

\usepackage{amsmath}

\begin{document}

\title{
    \textbf{Environemnt}
}
\author{Hannes Mann}
\date{2023-01-27}

\maketitle

\section*{Introduction}

This is a report for the second assignment in the course ID1019 Programming II.
In this assignment, a key-value database should be created using two different implementations: One using lists and the other using a binary search tree.

The implementation was done in Elixir. No build system has been set up at the time the report is written, but the script can be executed with \mintinline{elixir}{elixir main.exs {n}}.

The source code can be found at: \href{https://github.com/hannesmann/ID1019/tree/main/src/environment}{GitHub}.

\section*{Implementing the key-value database}

\subsection*{With lists}

The simplest way to implement a key-value database (or a map, as Elixir calls it) is to store pairs of keys and values in a list.
Adding is done by appending a new pair to the end of the list and lookup is done by searching through the list sequentially for the right key.

\begin{minted}[tabsize=2,fontsize=\footnotesize,breaklines]{elixir}
def add(map, key, value) do
  map ++ [{key, value}]
end

def lookup(map, key) do
  Enum.find_value(map, fn {k, v} -> k == key and v end)
end

def remove(map, key) do
  Enum.filter(map, fn {k, _} -> k != key end)
end
\end{minted}

While this approach is very simple to understand it doesn't scale well with larger lists as will be seen later in the benchmarks.

The list does not need to be sorted for this approach to work, but if it was sorted binary search could be used to speed up the lookup operation considerably.

\subsection*{With a binary search tree}

A binary search tree is a tree structure where every branch to the left of a node has smaller values and every branch to the right of a node has larger values.
This makes it a lot easier to search for keys as the structure is inherently sorted (though not balanced, depending on the operations executed on the list).

The tree was represented in Elixir as a tuple with the format \mintinline[breaklines]{elixir}{{key, value, left, right}}
where left and right are either \mintinline[breaklines]{cpp}{nil} or another tuple containing the nodes in a branch. The root node can also be \mintinline[breaklines]{cpp}{nil}.

Addition is the easiest to implement:
\begin{itemize}
	\item \textbf{If the root node is \mintinline[breaklines]{cpp}{nil}}: Create a new root node containing the first key-value pair.
	\item \textbf{If the key is smaller than the root node}: Walk to the left and repeat, treat the left branch as the new root node or add a new left branch (if none exists).
	\item \textbf{If the key is larger than the root node}: Walk to the right and repeat, treat the right branch as the new root node or add a new right branch (if none exists).
\end{itemize}

Lookup works in a very similar way but removal is more complicated. The problem occurs when we have found a node to remove and it has two branches.
If it had only one branch we would replace the current node with the branch node, but since there are two we have to preserve all their values.
The idea is to take the leftmost node in the right branch (which will be larger than all nodes in the left branch) and use that one as the new node.

\begin{minted}[tabsize=2,fontsize=\footnotesize,breaklines]{elixir}
  def leftmost({key, value, nil, rest}) do {key, value, rest} end
  def leftmost({key, value, left, right}) do
    {k, v, rest} = leftmost(left)
    {k, v, {key, value, rest, right}}
  end
\end{minted}

Binary search trees will complete the add, lookup and remove operation in $O(\log n)$ time on average, which makes them scale a lot better with larger lists.

\section*{Benchmarks}

Benchmarking is done by measuring the time it takes to complete $10000$ operations in an environment with $n$ keys. "map" is Elixir's built in map implementation.

\begin{table}[H]
\centering
\begin{tabular}{|c|c|c|c|}
\hline
\textbf{n} & \textbf{list} & \textbf{tree} & \textbf{map} \\
\hline
	16 & 60 ns & 40 ns & 40 ns \\
	32 & 60 ns & 50 ns & 40 ns \\
	64 & 100 ns & 60 ns & 40 ns \\
	128 & 200 ns & 70 ns & 50 ns \\
	256 & 350 ns & 80 ns & 60 ns \\
	512 & 820 ns & 90 ns & 80 ns \\
	1024 & 1500 ns & 120 ns & 60 ns \\
	2048 & 3500 ns & 110 ns & 70 ns \\
	4096 & 9800 ns & 120 ns & 80 ns \\
	8192 & 16000 ns & 150 ns & 90 ns \\
\hline
\end{tabular}
\caption{Execution time for adding a random key with $n$ keys.}
\label{tab:table1}
\end{table}

\begin{table}[H]
\centering
\begin{tabular}{|c|c|c|c|}
\hline
\textbf{n} & \textbf{list} & \textbf{tree} & \textbf{map} \\
\hline
  16 & 70 ns & 30 ns & 20 ns \\
  32 & 110 ns & 30 ns & 20 ns \\
  64 & 210 ns & 40 ns & 20 ns \\
  128 & 390 ns & 40 ns & 20 ns \\
  256 & 750 ns & 50 ns & 30 ns \\
  512 & 1600 ns & 50 ns & 30 ns \\
  1024 & 3000 ns & 60 ns & 30 ns \\
  2048 & 5900 ns & 70 ns & 30 ns \\
  4096 & 12000 ns & 70 ns & 30 ns \\
  8192 & 24000 ns & 90 ns & 30 ns \\
\hline
\end{tabular}
\caption{Execution time for finding a random key in $n$ keys.}
\label{tab:table2}
\end{table}

\begin{table}[H]
\centering
\begin{tabular}{|c|c|c|c|}
\hline
\textbf{n} & \textbf{list} & \textbf{tree} & \textbf{map} \\
\hline
  16 & 140 ns & 40 ns & 20 ns \\
  32 & 270 ns & 50 ns & 30 ns \\
  64 & 540 ns & 60 ns & 40 ns \\
  128 & 1000 ns & 70 ns & 40 ns \\
  256 & 2000 ns & 90 ns & 50 ns \\
  512 & 4300 ns & 90 ns & 50 ns \\
  1024 & 8700 ns & 110 ns & 60 ns \\
  2048 & 19000 ns & 120 ns & 60 ns \\
  4096 & 43000 ns & 130 ns & 60 ns \\
  8192 & 100000 ns & 160 ns & 90 ns \\
\hline
\end{tabular}
\caption{Execution time for removing a random key from $n$ keys.}
\label{tab:table3}
\end{table}

As expected, the list performs terribly, especially on remove operations.
This is because it first has to search through the list ($O(n)$)
and then it has to resize the list, potentially copying lots of items (worst case: $O(n)$).
The tree structure performs very well and is even fairly close to Elixir's built in implementation.

\end{document}
