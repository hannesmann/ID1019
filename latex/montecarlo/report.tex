\documentclass[a4paper,11pt]{article}

\usepackage[utf8]{inputenc}

\usepackage{minted}
\usepackage{hyperref}

\usepackage{amsmath}

\begin{document}

\title{
    \textbf{Monte Carlo and $\pi$}
}
\author{Hannes Mann}
\date{2023-02-24}

\maketitle

\section*{Introduction}

This is a report for the sixth assignment in the course ID1019 Programming II.
In this assignment, the Monte Carlo method should be used to approximate the value of $\pi$.

The implementation was done in Elixir. A bash script is included in the repository that can be used to run the program: \mintinline{text}{./run-project.sh montecarlo}.

The source code can be found at: \href{https://github.com/hannesmann/ID1019/tree/main/src/montecarlo}{GitHub}.

\section*{The Monte Carlo method}

A Monte Carlo method is a type of algorithm that works by repeatedly sampling random points as a way to approximate mathematical problems that would be difficult to solve by hand.
For example, we can approximate $\pi$ by thinking of the problem as throwing random darts at a circle of radius $r$ and seeing if the dart lands inside one of the quadrants of a circle, $r^2$ (that is, $r^2 > x^2 + y^2$).

In Elixir this would be expressed as a function that throws a dart at a random $(x, y)$ pair and returns if the dart landed inside or not:

\begin{minted}[tabsize=2,fontsize=\footnotesize,breaklines]{elixir}
def dart(r) do
  # Much faster than Enum.random
  x = :rand.uniform(r)
  y = :rand.uniform(r)

  r ** 2 > x ** 2 + y ** 2
end
\end{minted}

\subsection*{Simulating multiple rounds}

We define a round as throwing $k$ darts on a target and returning how many hit the target. This can be implemented recursively in Elixir like this:

\begin{minted}[tabsize=2,fontsize=\footnotesize,breaklines]{elixir}
def round(0, _, a) do a end
def round(k, r, a) do
  # If we hit the target
  case dart(r) do
    true -> round(k - 1, r, a + 1)
    false -> round(k - 1, r, a)
  end
end
\end{minted}

$r$ is the radius of the circle and $a$ is the number of darts that have hit so far.
The radius is important because we are not using floating-point math, so the range of random numbers needs to be large enough to be able to sample the circle accurately.

We can run the round function several times to determine $\pi$ with greater and greater accuracy.
The ratio of the number of darts hit to the number of darts thrown will start to approximate $\frac{\pi}{4}$
(since we're only checking one arch). Multiplying by $4$ will give us an approximation of $\pi$.

\begin{minted}[tabsize=2,fontsize=\footnotesize,breaklines]{elixir}
def rounds(0, _, t, _, a) do 4 * a/t end

# k = rounds left
# j = darts each round
# t = total thrown darts
# r = radius of circle
# a = total hit darts
def rounds(k, j, t, r, a) do
  a = round(j, r, a)
  t = t + j
  pi = 4 * a/t
  # print here
  rounds(k - 1, j, t, r, a)
end
\end{minted}

\subsection*{Testing the accuracy of our method}

To get an accurate result, the first thing we need to do is determine how large our radius should be.
I defined this as \mintinline{text}{trunc(:math.sqrt(134217728))} which is the largest integer that Elixir can efficiently represent on 32-bit machines and gives us a lot of range to perform the calculations accurately.

Running a thousand rounds with ten thousand darts for every round gives the following result:

\begin{minted}[tabsize=2,fontsize=\footnotesize,breaklines]{elixir}
round 100 - pi: 3.142520000000 abs error: 0.000927346410
round 200 - pi: 3.141410000000 abs error: 0.000182653590
round 300 - pi: 3.141084000000 abs error: 0.000508653590
round 400 - pi: 3.140919000000 abs error: 0.000673653590
round 500 - pi: 3.141127200000 abs error: 0.000465453590
round 600 - pi: 3.141130666667 abs error: 0.000461986923
round 700 - pi: 3.141254857143 abs error: 0.000337796447
round 800 - pi: 3.141349000000 abs error: 0.000243653590
round 900 - pi: 3.141942222222 abs error: 0.000349568632
round 1000 - pi: 3.141964000000 abs error: 0.000371346410
\end{minted}

Even after throwing $10000000$ darts we are still only accurate to the third decimal. This is enough to beat Archimedes but clearly this method needs a lot of samples to find accurate results.

We can try doubling the number of darts every round (so $2^1, 2^2, 2^3, $ etc...):

\begin{minted}[tabsize=2,fontsize=\footnotesize,breaklines]{elixir}
round (doubling) 1 - pi: 2.000000000000 abs error: 1.141592653590
round (doubling) 2 - pi: 2.000000000000 abs error: 1.141592653590
round (doubling) 3 - pi: 1.500000000000 abs error: 1.641592653590
round (doubling) 4 - pi: 2.250000000000 abs error: 0.891592653590
round (doubling) 5 - pi: 2.625000000000 abs error: 0.516592653590
...
round (doubling) 23 - pi: 3.141885280609 abs error: 0.000292627019
round (doubling) 24 - pi: 3.141875982285 abs error: 0.000283328695
round (doubling) 25 - pi: 3.141757369041 abs error: 0.000164715452
round (doubling) 26 - pi: 3.141656696796 abs error: 0.000064043207
\end{minted}

Even after throwing $2^{26} = 67108864$ darts we are still only accurate to the fourth decimal.
The method proposed in the course material (Leibniz formula) is accurate to the fourth decimal after only 1000 rounds.

\end{document}
