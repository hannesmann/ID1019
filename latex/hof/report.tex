\documentclass[a4paper,11pt]{article}

\usepackage[utf8]{inputenc}

\usepackage{minted}
\usepackage{hyperref}

\usepackage{amsmath}

\begin{document}

\title{
    \textbf{Higher-order functions}
}
\author{Hannes Mann}
\date{2023-02-10}

\maketitle

\section*{Introduction}

This is a report for the second higher-grade assignment in the course ID1019 Programming II.
In this assignment, the higher-order functions "map", "reduce" and "filter" should be implemented in Elixir along with some example functions that make use of them.

A bash script is included in the repository that can be used to run the program: \mintinline{text}{./run-project.sh hof}.

The source code can be found at: \href{https://github.com/hannesmann/ID1019/tree/main/src/hof}{GitHub}.

\section*{The Map function}

The first function that will be looked at is \mintinline{text}{map(list, f(elem))}. This function is considered a higher-order function because it takes another function as an argument.
It works by applying the function \mintinline{text}{f} on every element in the list and returning a new list with the transformed values.

We can start by looking at a function that could be trivially implemented with \mintinline{text}{map}:

\begin{minted}[tabsize=2,fontsize=\footnotesize,breaklines]{elixir}
def double([]) do [] end

def double([head | tail]) do
  [head * 2 | double(tail)]
end
\end{minted}

This function uses recursion to multiply all numbers in the list by two.
Depending on the programming language \mintinline{text}{double} could also modify the list in-place using a loop but in Elixir the recursive approach will work best.

Translating \mintinline{text}{double} to \mintinline{text}{map} is done by taking out the statement \mintinline{text}{head * 2}
and providing a function that applies the transformation to \mintinline{text}{map}. Iterating through the list will be done by \mintinline{text}{map}.

\begin{minted}[tabsize=2,fontsize=\footnotesize,breaklines]{elixir}
# Equivalent to double(list)
map(list, fn x -> x * 2)
\end{minted}

Since our original \mintinline{text}{double} function was really only a specialized version of \mintinline{text}{map} we can actually modify it slightly to implement \mintinline{text}{map} in Elixir.
We call the specified function with the syntax \mintinline{text}{f.(head)} to say that we want to treat the variable \mintinline{text}{f} as a function, rather than trying to call a function called \mintinline{text}{f} in the module.

\begin{minted}[tabsize=2,fontsize=\footnotesize,breaklines]{elixir}
# Call it apply_to_all, transform_all, map, etc...
def apply_to_all([], _) do [] end

def apply_to_all([head | tail], f) do
  [f.(head) | apply_to_all(tail)]
end
\end{minted}

The Map function is very useful because it allows easy parallelization of the program.
We know that every element in the list is independent, the "current" element does not depend on the "previous" element.
We could, for example, divide the list into $n$ separate lists (assuming we have $n$ threads on our computer) and run the specified function on these lists concurrently, only rejoining them at the very end.

\section*{The Filter function}

The second function that will be looked at is \mintinline[breaklines]{text}{filter(list, f(elem))}.
This is similar to Map but instead of transforming an element we now decide if the element should be removed from the list.
The function \mintinline[breaklines]{text}{f} will always return a boolean which is what determines if the element should be kept or removed.

An example of a function that can be implemented with \mintinline[breaklines]{text}{filter} is only keeping the odd numbers in a list:

\begin{minted}[tabsize=2,fontsize=\footnotesize,breaklines]{elixir}
def odd([]) do [] end
def odd([head | tail]) do
  case head do
    head when rem(head, 2) == 1 -> [head | odd(tail)]
    _ -> odd(tail)
  end
end
\end{minted}

Here the condition is \mintinline[breaklines]{elixir}{rem(head,2) == 1} so using \mintinline[breaklines]{text}{filter} would give us:

\begin{minted}[tabsize=2,fontsize=\footnotesize,breaklines]{elixir}
# Equivalent to odd(list)
filter(list, fn x -> rem(x, 2) == 1)
\end{minted}

Same as with \mintinline[breaklines]{text}{map} our function is a specialized version of \mintinline[breaklines]{text}{filter}, so making some changes gives us this:

\begin{minted}[tabsize=2,fontsize=\footnotesize,breaklines]{elixir}
def filter([], _) do [] end
def filter([head | tail], f) do
  if f.(head) do [head | filter(tail, f)] else filter(tail, f) end
end
\end{minted}

Filter is useful for the same reason as Map, the condition for each element doesn't depend on any other element so it allows easy parallelization.

\section*{The Reduce function}

The last function that will be looked at is \mintinline[breaklines]{text}{reduce(list, base, f(elem, acc))}.
This function repeatedly calls \mintinline[breaklines]{text}{f} on the elements in a list, with \mintinline[breaklines]{text}{acc} being the value of the the previous function invocation.
If this is the first function invocation, the base value is used as a starting point for the accumulator.

One of the most obvious examples of where this would be useful is taking the sum of all the values in a list:

\begin{minted}[tabsize=2,fontsize=\footnotesize,breaklines]{elixir}
def sum([]) do 0 end
def sum([head | tail]) do head + sum(tail) end
\end{minted}

This function actually shows one of the problems we have to solve with \mintinline[breaklines]{text}{reduce}: Do we add the head at the beginning or the end of the statement?
For addition this doesn't actually make a difference, since it's commutative, but if we wrote for example \mintinline{elixir}{head ** sum(tail)} it would return a different value if we
changed the order.

This gives us two different ways to solve the problem: Folding from the right and folding from the left. Folding from the right means we start at the last element and accumulate from there:

\begin{minted}[tabsize=2,fontsize=\footnotesize,breaklines]{elixir}
def fold_right([], base, _) do base end
def fold_right([head | tail], base, f) do
  f.(head, fold_right(tail, base, f))
end
\end{minted}

Folding from the left means we start at the first element but it's a little trickier to implement.
What we do is call the function on the first element with our base value.
Then, we treat the result of that function as the new base value and reduce again on the remaining part of the list:

\begin{minted}[tabsize=2,fontsize=\footnotesize,breaklines]{elixir}
def fold_left([], base, _) do base end
def fold_left([head | tail], base, f) do
  fold_left(tail, f.(head, base), f)
end
\end{minted}

Reduce is very versatile. For example, we can implement Map by starting out with an empty list and accumulating a final list with the transformed values,
or we can implement Filter by starting out with an empty list and accumulating a final list with only the values that weren't filtered.

\end{document}
