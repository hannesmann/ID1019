\documentclass[a4paper,11pt]{article}

\usepackage[utf8]{inputenc}

\usepackage{minted}
\usepackage{hyperref}

\usepackage{amsmath}

\begin{document}

\title{
    \textbf{A math evaluator in Elixir}
}
\author{Hannes Mann}
\date{2023-02-03}

\maketitle

\section*{Introduction}

This is a report for the third assignment in the course ID1019 Programming II.
In this assignment, a program should be created that can interpret and evaluate a math expression. Variables will be defined by the user and resolved during evaluation.

The implementation was done in Elixir. No build system has been set up but the script can be executed with \mintinline{elixir}{elixir main.exs}.

The source code can be found at: \href{https://github.com/hannesmann/ID1019/tree/main/src/math}{GitHub}.

\section*{Implementing the evaluator}

\subsection*{Environment}

The environment contains a mapping of variables to their assigned values. The user provides either a map or an array with key-value pairs which are resolved by the evaluator when a variable is found.

For simplicity and speed the included map module in Elixir was used to implement the environment.
The environment provides an interface to allow for alternative backends (like a list or the tree from the previous assignment), it's not bound to one implementation.

\begin{minted}[tabsize=2,fontsize=\footnotesize,breaklines]{elixir}
defmodule Environment do
  def new() do %{} end
  def new(bindings) when is_map(bindings) do bindings end

  def new(bindings) when is_list(bindings) do
    Enum.reduce(bindings, %{}, fn ({k, v}, map) -> Map.put(map, k, v) end)
  end

  def find(env, key) do env[key] end
end
\end{minted}

\subsection*{Expressions}

Expressions are represented by a tuple with two or three values: An atom specifying the type of the expression and the arguments for the value or operation.

Data values can be either integers, rational numbers or variables.

\begin{minted}[tabsize=2,fontsize=\footnotesize,breaklines]{elixir}
@type literal() ::
  { :num, number() } |
  { :q, number(), number() } |
  { :var, atom() }
\end{minted}

The available operations are addition, subtraction, multiplication and division. These expressions, when evaluated, resolve into different expressions which are data values.

\begin{minted}[tabsize=2,fontsize=\footnotesize,breaklines]{elixir}
@type expr() ::
  {:add, expr(), expr()} |
  {:sub, expr(), expr()} |
  {:mul, expr(), expr()} |
  {:div, expr(), expr()} |
  literal()
\end{minted}

\subsection*{Pattern matching}

The evaluator uses pattern matching to recursively evaluate an expression and its subexpressions.
To make the implementation as simple as possible, all integers and variables are converted into rational numbers.
A variable will always resolve into an integer or rational number, and an integer can always be expanded into a rational number of the form $n \Longleftrightarrow \frac{n}{1}$.

\begin{minted}[tabsize=2,fontsize=\footnotesize,breaklines]{elixir}
def eval({:num, n}, _) do {:q, n, 1} end
def eval({:var, k}, env) do
  eval(Environment.find(env, k), env)
end
\end{minted}

Since we know all expressions will eventually be converted into rational numbers, we only need to provide a single implementation of every operation and we know the result of evaluating a complete math expression will be a single rational number.

\begin{minted}[tabsize=2,fontsize=\footnotesize,breaklines]{elixir}
def eval({:operation, {:q, a, b}, {:q, c, d}}, env) do
  # Do operation on rational numbers
end

def eval({:operation, a, b}, env) do
  # The function only needs to "pass through" to the real implementation
  eval({:operation, eval(a, env), eval(b, env)}, env)
end
\end{minted}

\subsection*{Operations}

\begin{itemize}
	\item \textbf{Addition}: Implemented by the formula: $\frac{a}{b} + \frac{c}{d} = \frac{(a*d)+(c*b)}{b*d}$.
  This will of course create intermediate results with large numbers (example:  $\frac{50}{10} + \frac{24}{4} = \frac{440}{40}$)
  but this doesn't matter since the final result will be simplifed before being shown to the user.
  \item \textbf{Subtraction}: Implemented with addition: $\frac{a}{b} - \frac{c}{d} = \frac{a}{b} + \frac{-c}{d}$.
  \item \textbf{Multiplication}: Implemented by the formula: $\frac{a}{b} * \frac{c}{d} = \frac{a*c}{b*d}$.
  \item \textbf{Division}: Implemented with multiplication: $\frac{\frac{a}{b}}{\frac{c}{d}} = \frac{a}{b} * \frac{d}{c}$.
\end{itemize}

\subsection*{Simplification}

The final step in evaluation is to take the finished result and simplify it (if necessary).
This is implemented by first taking the final rational $p/q$ and finding the greatest common divisor of $p$ and $q$.
Both terms are then divided by the GCD. The program uses \mintinline{elixir}{Integer.gcd} from the Elixir standard library but this function
could also easily be manually implemented with Euclid's algorithm or some other clever algorithm.

\begin{minted}[tabsize=2,fontsize=\footnotesize,breaklines]{elixir}
gcd = Integer.gcd(a, b)
# Kernel.div necessary for integer division
reduced = {:q, Kernel.div(a, gcd), Kernel.div(b, gcd)}
\end{minted}

After the number has been divided as much as possible it's determinted if it can be represented as a integer or not.
If the denominator is 1, the result is an integer, otherwise the result is a rational number.

\begin{minted}[tabsize=2,fontsize=\footnotesize,breaklines]{elixir}
case reduced do
  {_, n, 1} -> {:num, n}
  _ -> reduced
end
\end{minted}

\section*{Results}

The test program will print out the result of evaluating three expressions:

\begin{minted}[tabsize=2,fontsize=\footnotesize,breaklines]{elixir}
# Expression
{:div, {:mul, {:num, 5}, {:num, 2}}, {:num, 5645}}
# Environment
%{}
# Result
2/1129

# Expression
{:mul, {:num, 5}, {:var, :x}}
# Environment
%{x: {:num, 10}}
# Result
50

# Example from course material
# Expression
{:add, {:add, {:mul, {:num, 2}, {:var, :x}}, {:num, 3}}, {:q, 1, 2}}
# Environment
%{x: {:num, 80}}
# Result
327/2
\end{minted}

\end{document}
