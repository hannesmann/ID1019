\documentclass[a4paper,11pt]{article}

\usepackage[utf8]{inputenc}

\usepackage{minted}
\usepackage{hyperref}

\usepackage{amsmath}

\begin{document}

\title{
    \textbf{Train shunting}
}
\author{Hannes Mann}
\date{2023-03-03}

\maketitle

\section*{Introduction}

This is a report for the seventh assignment in the course ID1019 Programming II.
In this assignment, a program should be created that can compute the moves required to shunt a train into a specific position.

The implementation was done in Elixir. A bash script is included in the repository that can be used to run the program: \mintinline[breaklines]{text}{./run-project.sh train}.

The source code can be found at: \href{https://github.com/hannesmann/ID1019/tree/main/src/train}{GitHub}.

\section*{The problem}

Train shunting is the process of sorting different wagons of a train into a specific order.
We have three tracks, one "main" track where the train should be located at the end and two auxiliary tracks we can use when sorting.

The tracks are represented by a tuple in Elixir of the format \mintinline[breaklines]{text}{{main, one, two}} where all elements are a train of the format \mintinline[breaklines]{text}{[:a, :b, :c, ...]}.
Every atom in the list is a wagon and these wagons are what will be rearranged to achieve the specified order at the end.

\subsection*{List manipulation}

To be able to sort the train we need to be able to manipulate the lists that are given to us.
In this exercise, no built in Elixir modules should be used, so instead these functions were defined in a module called \mintinline[breaklines]{text}{Train}:

\begin{minted}[tabsize=2,fontsize=\footnotesize,breaklines]{elixir}
take(list, n)
# Take n elements from the beginning of the list
drop(list, n)
# Drop n elements from the beginning of the list
append(a, b)
# Append list b to list a
member(list, y)
# Returns true if y is an element of list
position(list, y)
# Returns the position of y in list (starting from 1), or :nomatch
split(list, y)
# Returns a tuple of {before, after} split between element y
main(list, n)
# Takes up to n elements from the list
# Returns a tuple of {k, remain, take} where k is the number of "missing" wagons in take
\end{minted}

The function \mintinline[breaklines]{text}{main} was implemented using recursion and checking if it was possible to keep taking elements from the list or not.
If no more elements can be taken, the first element is added to \mintinline[breaklines]{text}{remain}, otherwise the element is added to \mintinline[breaklines]{text}{take}.

\subsection*{Applying moves}

Moves are applied with a function called \mintinline[breaklines]{text}{single}. This uses \mintinline[breaklines]{text}{Train.main} to take from either main, one or two.

If the argument is positive, elements are moved from main to one or two, otherwise elements are moved from one or two to main.

\subsection*{Finding the best moves}

The best moves are found with a function called \mintinline[breaklines]{text}{find}. This moves from one and two, then back to main.
This creates a lot of redundant moves so an optimized function \mintinline[breaklines]{text}{few} was created that looks like this:

\begin{minted}[tabsize=2,fontsize=\footnotesize,breaklines]{elixir}
def few([], []) do [] end
def few(xs, [y | ys]) do
  {hs, ts} = Train.split(xs, y)

  if length(hs) == 0 do
    Train.append([], few(Train.append(hs, ts), ys))
  else
    moves = [{:one, length(ts) + 1}, {:two, length(hs)}, {:one, -length(ts) - 1}, {:two, -length(hs)}]
    Train.append(moves, few(Train.append(hs, ts), ys))
  end
end
\end{minted}

If we only need to move one section back and forth, it has no effect, so we can skip appending moves in that case.

\subsection*{Compressing the result further}

We can optimize even more by applying some simple rules:

\begin{minted}[tabsize=2,fontsize=\footnotesize,breaklines]{elixir}
def rules([]) do [] end
def rules([head | tail]) do
  case head do
    {_, 0} -> rules(tail)
    {:one, n} ->
      case tail do
        [{:one, m} | rest] -> [{:one, n + m} | rules(rest)]
        _ -> [head | rules(tail)]
      end
    {:two, n} ->
      case tail do
        [{:two, m} | rest] -> [{:two, n + m} | rules(rest)]
        _ -> [head | rules(tail)]
      end
  end
end

def compress(ms) do
  ns = rules(ms)
  if ns == ms do
    ms
  else
    compress(ns)
  end
end
\end{minted}

Rules are applied recursively until no more changes can be made.

\section*{Result}

Testing on the examples in the course material gives the following results:

\begin{minted}[tabsize=2,fontsize=\footnotesize,breaklines]{elixir}
test_train: [:b, :c, :a, :d]
Train.take(test_train, 2): [:b, :c]
Train.drop(test_train, 2): [:a, :d]
Train.append(test_train, [:e, :f]): [:b, :c, :a, :d, :e, :f]
Train.member(test_train, :b): true
Train.member(test_train, :x): false
Train.position(test_train, :b): 1
Train.position(test_train, :a): 3
Train.position(test_train, :x): :nomatch
Train.split(test_train, :b): {[], [:c, :a, :d]}
Train.split(test_train, :a): {[:b, :c], [:d]}
Train.split(test_train, :z): {[:b, :c, :a, :d], []}
Train.main([:a, :b, :c, :d], 3): {0, [:a], [:b, :c, :d]}
Train.main([:a, :b, :c, :d], 3): {1, [], [:a, :b, :c, :d]}

Moves.sequence(...): [
  {[:a, :b], [], []},
  {[:a], [:b], []},
  {[], [:b], [:a]},
  {[:b], [], [:a]},
  {[:b, :a], [], []}
]

Shunt.find(...): [one: 1, two: 1, one: -1, two: -1, one: 1, two: 0, one: -1, two: 0]
Shunt.few(...): [one: 1, two: 1, one: -1, two: -1]

Shunt.compress([{:two,-1},{:one,1},{:one,-1},{:two,1}]): []
\end{minted}

\end{document}
